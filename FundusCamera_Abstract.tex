%% LyX 1.6.4 created this file.  For more info, see http://www.lyx.org/.
%% Do not edit unless you really know what you are doing.
\documentclass[10pt,a4paper,english]{amsart}
\usepackage[T1]{fontenc}
\usepackage[utf8]{inputenc}
%\usepackage{endnotes}
\usepackage{units}
%\usepackage{multirow}
\usepackage{amstext}
\usepackage{amsmath}
\usepackage{amssymb}
\usepackage{amsfonts}
\usepackage{enumerate}
\usepackage{cite}
%\usepackage{natbib}
\usepackage{amsthm}
\usepackage{array,arydshln}
\usepackage[pdftex]{graphicx}
\usepackage{rotating}
\usepackage{ifpdf}
%\usepackage{epsfig}
\usepackage[all]{xy}
\usepackage{latexsym}
\usepackage[hidelinks]{hyperref}
\usepackage{color}
\usepackage{eucal}
\usepackage{mathrsfs}
\usepackage{kotex}
\usepackage{slashbox}
%\usepackage{ulem}
%\usepackage{hfont}


\makeatletter
%%%%%%%%%%%%%%%%%%%%%%%%%%%%%% Textclass specific LaTeX commands.
\numberwithin{equation}{section} %% Comment out for sequentially-numbered
\numberwithin{figure}{section} %% Comment out for sequentially-numbered
\numberwithin{table}{section}
\let\footnote=\endnote
\theoremstyle{plain}
\newtheorem{thm}{Theorem}[section]
\newtheorem{theorem}[thm]{Theorem}
\newtheorem{Theorem}[thm]{Theorem}
\theoremstyle{definition}
\newtheorem{defn}[thm]{Definition}
\newtheorem{Def}[thm]{Definition}
\newtheorem{definition}[thm]{Definition}
\newtheorem{exam}[thm]{Example}
\newtheorem{algo}[thm]{Algorithm}
%  \theoremstyle{plain}
\newtheorem{assumption}[thm]{Assumption}
\theoremstyle{plain}
\newtheorem{lem}[thm]{Lemma}
\newtheorem{lemma}[thm]{Lemma}
\theoremstyle{plain}
\newtheorem{cor}[thm]{Corollary}
\newtheorem{corollary}[thm]{Corollary}
\theoremstyle{plain}
\newtheorem{rmk}[thm]{Remark}
\newtheorem{rem}[thm]{Remark}

\def\norm#1{\|#1\|}

\newcommand\numberthis{\addtocounter{equation}{1}\tag{\theequation}}

%\newcommand{\norm}[1]{\|#1\|}
\def\norm#1{\|#1\|}
\def\normm#1#2{\|#1\|_{#2}}
\def\normF#1{\|#1\|_{F}}
\def\Proof{{\bf Proof.\enspace}}
\def\vec{{\sf vec}}
%\def\unvec{\mathrm{unvec}}
\def\tr{\mathrm{tr}}
%\def\tr{\textrm{tr}}
\def\bmatrix#1{\left[\begin{matrix}#1\end{matrix}\right]}
\def\pmatrix#1{\left(\begin{matrix}#1\end{matrix}\right)}
\def\RR{\mathbb{R}}
\def\NN{\mathbb{N}}
\def\CC{\mathbb{C}}
\def\nbyn{n\times n}
\def\mbyn{m\times n}
\def\mbym{m\times m}
\def\pbyq{p\times q}
\def\nnbynn{n^{2}\times n^{2}}
\def\mbf#1{\mathbf{#1}}
\def\mrm#1{\mathrm{#1}}
\def\bpi{\boldsymbol{\pi}}
\def\a{\alpha}
\def\b{\beta}
\def\d{\delta}
\def\e{\varepsilon}
\def\l{\lambda}
\def\D{\mathfrak{D}}
\def\F{\mathcal{F}}
\def\G{\mathcal{G}}
\def\Q{\mathcal{Q}}
\def\M{\mathcal{M}}
\def\N{\mathcal{N}}
\def\P{\mathcal{P}}
\def\X{\mathfrak{X}}
\def\proj{\mathbf{P}}
\def\pjn{\mathbf{P}_{\mathcal{N}}}
\def\pjm{\mathbf{P}_{\mathcal{M}}}
\def\tXi{\widetilde{X}_{i}}
\def\tXii{\widetilde{X}_{i+1}}
\def\dpm#1{\begin{displaymath}#1\end{displaymath}}
\def\bdm{\begin{displaymath}}
\def\edm{\end{displaymath}}
\def\dtyl{\displaystyle}
\def\ones#1{\mathbf{1}_{#1}}
%\def\onesn{\mathbf{1}_{n \times n}}

\definecolor{gray}{rgb}{.7,.7,.7}

\makeatother

\begin{document}
	
\title{Introducing of Fundus Cameras based on Convolutional Neural Network}
\author{Sang-hyup Seo}
\address{Sang-hyup Seo\\Busan Center for MM, National Institute for Mathematical Science, 49241, Gudeok-ro 187, Busan, Republic of Korea}
\email{saibie1677@nims.re.kr}
\date{\today}

\begin{abstract}
%	We consider the Newton iteration for a matrix polynomial equation which arises in stochastic problem.
%	In this paper, we show that the elementwise minimal nonnegative solution of the matrix polynomial equation can be obtained using Newton's method if the equation satisfies the sufficient condition, and
%	the convergence rate of the iteration is quadratic if the solution is simple.
%	Moreover, we show that the convergence rate is at least linear if the solution is non-simple,
%	but we can apply a modified Newton method whose iteration number is less than the pure Newton iteration number.
%	Finally, we give a numerical experiment which is related with our issue.
\end{abstract}

%\keywords{matrix polynomial equation, positive solution, nonnegative solution, $M$-matrix, Newton's method, line search, acceleration of a method, nearly non-simple}

%\subjclass[2010]{65H10}

% \thanks{$\dagger$Corresponding Author}

\maketitle

\section{Introduction}\label{sec:intro}



\bibliographystyle{plain}
\bibliography{SHSeo}

\end{document}