\documentclass[11pt]{ksiamproc_h}
\usepackage{hangul,times}
\usepackage{float,graphicx,psfig}
\usepackage{amssymb,latexsym,amsmath}
\pagestyle{empty}   %%%
\begin{document}
\begin{frontmatter}
\title\protect{\begin{center} \LARGE\bf %
요약본 원고 작성 지침 \\[1ex]                           %%  한글제목
\Large\bf                        %%
INSTRUCTIONS TO AUTHORS FOR THE PREPARATION OF PAPERS   %% 영문제목
\end{center}}%


\author[]{}{고현무}             %%% 제일 저자  [hmk]는 제1저자와 이해성의 주소를 위한 라벨입니다.
\author[hmk]{}{, 이해성}        %%% 공동
\author[hmk]{}{, 백두산}        %%% 공동저자
\author[bds]{}{}                %%% 꼭 필요합니다. [bds]는 공동저자 백두산의 reference입니다.

\medskip
\address[hmk]{}{\vskip-18pt \hskip12pt\ %
서울대학교 지구환경시스템공학부, 서울 151-742   % 여기에 주소를 넣으시고 위의 vskip과  hskip은 그대로 두시오.
}%  %%% 다음에 반드시 빈 줄이 있어야 합니다.

\address[bds]{}{\vskip-18pt \hskip12pt\ %
부산대학교 수학과, 부산 617-736  %여기에 주소를 넣으시오.
}%

\medskip %
\centering{교신 저자$\;:\;$고현무,$\;$ e-mail@Address.xxx
}%

\begin{abstract}
이 문서는 한국산업응용수학회(KSIAM) 춘계학술대회 및 추계학술대회에
제출하는 논문 원고를 작성하는 지침으로  제출 원고를 작성하기 위한
template로 사용하실 수 있습니다.  원고는 반드시 pdf 파일로
제출하셔야 하며, pdf 형식이 아닌 파일은 upload 하실 수 없습니다. 이
지침을 따르지 않은 원고는 형식 수정을 위하여 교신 저자에게
반환됩니다.  발표할 논문을 한글{\LaTeX}을 이용하여 준비하실 경우는
클래스 파일인 \verb|ksiamproc_h.cls|를 사용하시고 이 template를
이용하여 작성하십시오.  자세한 사항은 학술대회 홈페이지의 안내를
참고하시기 바랍니다. %
%원고 전체에서 한글 폰트는 바탕체를 사용하시고 영문 폰트로는 times
%new roman 체를 사용하셔서 본문을 작성하셔야 합니다. 폰트의 크기는 11
%포인트이며, single space를 적용하시고 양쪽 맞춤을 사용합니다.  이
%지침에서 빈 줄은 11 포인트. 바탕체를 사용할 경우의 빈 줄을
%의미합니다. 논문 제목은 한글 제목을 먼저 표기하시고 그 다음에 영문
%제목을 표기 하십시오. 한글 제목과 영문 제목은 이 template에서 정한
%크기를 사용하십시오. 영문제목과 한글제목사이에는 빈 줄을
%삽입하십시오. 위에서 보인 바와 같이 영문제목 다음에 빈 줄을
%삽입하시고 저자 이름을 12 포인트 크기로 표기하십시오.  저자 이름
%다음에는 각 저자의 소속 기관과 소속 기관이 위치한 도시(혹은 도)와
%우편번호를 쓰십시오. 저자의 소속 기관 다음에는 빈 줄을 삽입하시고
%교신 저자의 이름과 e-mail 주소를 쓰십시오.
원고의 길이는 초청 논문 경우에는 6 페이지이며, 일반 논문은 최소 2
페이지, 최대 4 페이지 입니다.  만일 작성하신 논문이 3 페이지일
경우에는 마지막에 빈 페이지를 삽입하여 총 4 페이지를 제출하여
주십시오.
\end{abstract}

\end{frontmatter}

\section*{일차 제목}\noindent%
일차 제목은 볼드체를 사용하시고 한글일 경우 11 포인트, 영문일
경우에는 12 포인트의 대문자를 사용하시고 일차 제목의 아래 위에
빈줄을 삽입하십시오. 일차 제목은 중앙 정렬합니다.
\subsubsection*{이차 제목}
이차 제목은 한글, 영문 모두 11 포인트를 사용하십시오. 영문일 경우
이차 제목의 각 단어의 첫 글자를 대문자로 표기하십시오. 이차 제목
위에는 빈 줄을 삽입하시고, 아래에는 빈 줄을 삽입하지 않습니다. 이차
제목은 밀어 맞춤하지 않고 왼쪽 정렬하십시오. 각 문단의 첫 줄은 0.7cm
밀어 맞춤을 합니다. 각 문단 전에는 빈 줄을 삽입하지 않습니다. 빈 줄
다음에 나오는 첫번째 문단의 첫 번째 줄은 밀어 맞춤하지 않습니다.
\section*{수식, 표, 그림}\noindent%
모든 수식은 중앙 정렬하며, 각 수식은 원고에 나오는 순서에 따라
아라비아 숫자로 번호를 매겨야 합니다.  수식의 번호는 오른쪽 정렬을
합니다.
\begin{equation}\label{eq:1}
\mathbf{Ma}+\mathbf{C(x)v}+\mathbf{K(x)u}=\mathbf{P}(t)
\end{equation}
표와 그림은 반드시 본문 내에 위치하여야 하며, 가급적
각 페이지의 위쪽과 아래쪽에 위치하게 편집하여 주십시오.
 표와 그림은 본문에서 인용되는 순서대로 각각 일련 번호를 아라비아
 숫자로 매겨야 합니다.  표와 그림은 본문에서 각각 표 1.,
 그림 1과 같이 인용합니다.  표의 제목은 표의 위에 위치하며, 그림의 제목은 그림
 아래에 위치합니다.  표의 제목과 표 사이에는 빈줄을 삽입하지 않고,
 그림의 제목과 그림 사이에는 빈 줄을 삽입하십시오.

\begin{table}[ht]
\caption{표 제목}
\end{table}
\noindent\framebox[15.cm]{표 혹은 그림}

\begin{figure}[ht]
%\centerline{\psfig{file=Trig/p6.eps,width=6.5cm}}
\caption{\label{fig:p6} 그림 제목}
%\includegraphics
\end{figure}


\section*{참고문헌}\noindent%
참고 문헌의 본문에서 인용된 순서에 따라 아라비아 숫자로 번호를 매겨 나열합니다.
 참고 문헌은 리스트는 원고의 맨 마지막에 위치하며,
 본문에서는 꺽쇠 괄호 안에서 참고 문헌의 번호를 인용합니다 [1, 2, 3].
 참고 문헌 리스트의 형식은 다음과 같습니다.
\\

\noindent (1) 단행본

\noindent1.  Hughes, T. J. R., {\it The Finite Element Method,
Linear Static and Dynamic Finite Element  Analysis},
Prentice-Hall, Engelwood Cliffs, NJ, 1987.

\bigskip

\noindent (2) 논문집 논문

\noindent2.  Koh, H. M., Lee, H. S. and Haber, R. B., ``Dynamic
crack propagation analysis using Eulerian-Lagrangian kinematic
descriptions'', {\it Computational Mechanics}, Vol. 3, 1988, pp.
141-155.

\bigskip
\noindent (3) 단행본의 장

\noindent3.  Riedel, H., ``Nucleation of Creep Cavities/Basic
Theories", Chapter 7, {\it Fracture at High Temperatures},
Springer-Verlag, Berlin, 1987.

\bigskip
\noindent (4) 학술대회 논문집 논문

\noindent4.  Lee, H.S. and Koh, H.M., ``A Moving-Grid Finite
Element Method for the Prediction of Dynamic Crack Propagation in
Brittle Materials", {\it Proc. of the Second International
Conference on Computer Aided Assessment and Control of Localized
Damage}, Vol. 2, pp 463-480, Southampton U.K., July 1992.



\begin{thebibliography}{11}
\bibitem {hugh}{Hughes, T. J. R.},
\newblock{\em The Finite Element Method, Linear Static and Dynamic Finite
Element  Analysis}, Prentice-Hall, Engelwood Cliffs, NJ, 1987.

\bibitem {KLH}{ Koh, H. M., Lee, H. S. and Haber, R. B.},
{``Dynamic crack propagation analysis using Eulerian-Lagrangian
kinematic descriptions''}, {\em Computational Mechanics}, Vol. 3,
1988, pp. 141-155.

\bibitem {riedel}{Riedel, H.},
{``Nucleation of Creep Cavities/Basic Theories"},  Chapter 7, {\it
Fracture at High Temperatures}, Springer-Verlag, Berlin, 1987.

\bibitem {LK}{Lee, H.S. and Koh, H.M.},
{``A Moving-Grid Finite Element Method for the Prediction of
Dynamic Crack Propagation in Brittle Materials"}, {\it Proc. of
the Second International Conference on Computer Aided Assessment
and Control of Localized Damage}, Vol. 2, pp 463-480, Southampton
U.K., July 1992.
\end{thebibliography}
\end{document}
