\documentclass[12pt]{ksiamproc}
\usepackage{times}
\usepackage{float,graphicx}
\usepackage{amssymb,latexsym,amsmath}
\renewcommand\baselinestretch{1}
%%%
%%% pdf 파일 작성시에는 pdflatex, latex+dvips+ps2pdf, latex+dvipdfm, latex+dvipdfmx
%%%  중에서 적절한 명령을 사용하시면 됩니다.
%%%
%%% You may use one of the following command to generate a pdf-file
%%% pdflatex, latex+dvips+ps2pdf, latex+dvipdfm, latex+dvipdfmx
%%%
%%%%
%% If you must include aknowledgement for support, insert
%% the followng section just before the reference section
%%
%% \section*{Aknowledgement}\noindent%
%% This work .....
%%
%%
\begin{document}
	\pagestyle{empty}
	\begin{frontmatter}
		%%
		%%%  The title should be in 18-point bold face with uppercase
		%%%%  제목의 모든 문자는 18-pt boldface 대문자를 사용하여야 합니다.
		%%%%%
		\title{INSTRUCTIONS TO AUTHORS FOR THE PREPARATION OF PAPERS}
		%%
		%%
		%% 저자를 리스트하는 형식(\author)에서  콤마와(,) and에 주의하시기 바랍니다.
		%% Be aware of comma(,) and \and in the \author command
		%%
		%%
		\author[hmk]{Hyun-Moo KOH}
		\author[hmk]{,~ Hae Sung LEE}
		\author[bds]{\and~ Doo San BAIK}
		%%
		%%
		%아래 \address와 위의 \author 사이에 blank line이 있어야 합니다.
		%%%% There should be a blank line between \author and \address
		%%
		
		\medskip
		%%%%
		\address[hmk]{Department of Civil Engineering, Seoul National University, Seoul 151-742, KOREA}
		\address[bds]{Department of Civil Engineering, Shilla University, Busan 617-736, KOREA}
		%
		% 이 바로 아래는 blank line이 있어야 합니다. there should come a blank line after this line
		
		\medskip
		\centering{Corresponding Author$\;:\;$Hyun-Moo KOH,
			e-mail@Address.xxx}
		%
		\begin{abstract}
			The following guidelines are to provide general rules for the
			preparation of the manuscripts to be submitted to KSIAM spring and
			fall conferences.  This document may be used as the template of your
			manuscript.  Manuscripts should be submitted in pdf format.
			Manuscripts other than pdf format are not uploadable.  Manuscripts
			that do not follow this guideline will be returned to corresponding
			authors for the correction of format.  Please justify your text and
			use 12 point times new roman fonts and single space in the entire
			manuscript.  In this guideline, a blank space denotes a singe spaced
			blank line for 12 point times new roman font.  For the title of the
			paper, authors' names and their affiliations, please use 18 point
			times new roman font in bold face, 12 point times new roman font and
			12 point times new roman font in italic face, respectively, as shown
			above. After authors' affiliation list, specify the corresponding
			author's name and e-mail address. Please leave one blank space
			between the title, authors' names, affiliations and information on
			corresponding author.  The lengths of papers for invited papers are
			6 pages, and those for normal papers are 2 pages minimum, 4 pages
			maximum.  In case the length of your manuscript is 3 pages, please
			insert a blank page at the end of your manuscript.
			
		\end{abstract}
	\end{frontmatter}
	%%
	%%
	%% \section과 \subsubsection에 *가 있어야 합니다.
	%% section과 subsubsection에 numbering을 하지 않습니다.
	%% \section 다음에 \subsection이 아니라 \subsubsection임에 주의 하시기 바랍니다.
	%% 반드시 section 다음에 \noindent를 명령해야 합니다.
	%% \subsubsection에서 각 단어의 첫 글자는 대문자이어야 합니다.
	%%
	%% \section and \subsubsection must have the asterisk(*) symbol as shown below.
	%%  \section must have \noindent command at the end of the command
	%%   as shown below
	%%
	
	\section*{Primary heading}\noindent%
	Primary headings are in 12-point bold face with upper cases, and
	center-aligned.  A main heading should have one line space above
	and below it.
	%
	%%
	%% Each word in \subsubsection must start with the capital letter
	%%
	%
	\subsubsection*{Secondary Heading} The first characters of all
	words in a secondary heading are written in upper cases, and left
	adjusted. A secondary heading should have one line of space above
	it but no space below it.
	
	Paragraphs should be indented 0.7 cm or 5 character spaces,
	without any extra space above.  Do not indent headings and the
	first line after blank space.
	%
	%
	\section*{EQUATIONS, TABLES AND FIGURES}\noindent%
	All forms of equations have to be centered and numbered
	consecutively with Arabic numerals as they appear in the text of
	the paper. The equation would be presented as follows:
	\begin{equation}\label{eq:1}
		\mathbf{Ma}+\mathbf{C(x)v}+\mathbf{K(x)u}=\mathbf{P}(t)
	\end{equation}
	Tables and figures must be integrated with the text and numbered
	consecutively with Arabic numerals in the order in which reference
	is made to them in the text of the paper.  The table or figure
	caption would be referred to in the text as Table 1 or~ Fig. 1,
	respectively, and be presented as follows.
	
	\begin{table}[ht]
		\caption{The first table in the paper}
		\begin{tabular}{|ccc|}\hline
			%A&B&C\\ \hline
			%0.1&0.2&1\\ \hline
			&&\\
			\phantom{MMMMMMMMMMMMMMMM}& Table or~ Figure&\phantom{MMMMMMMMMMMMMMMM}\\
			&&\\ \hline
		\end{tabular}
	\end{table}
	
	\begin{figure}[ht]
		%\centerline{\psfig{file=fig1.eps,width=6.5cm}}
		%\includegraphics[width=4.7cm,angle=270]{ellipsoid.ps}
		\caption{\label{fig:fig1} The first figure in the paper}
	\end{figure}
	%%
	%% If you want to include acknowledgement uncomment the following
	%% and supply appropriate paragraph
	%%\section*{Acknowledgement}\noindent%
	%% This work .....
	%%%
	\section*{References}\noindent%
	%%
	%% 참고문헌은 아래 마지막 부분의 \thebibliography environment를 이용합니다.
	%% 이 섹션은 단지 설명을 위한 것 입니다.
	%%
	%% This section is just for instruction how to prepare reference setion.
	%% Use \thebibliography environment for reference as shown below after this section
	%%
	%%
	References are to be listed at the end of the paper in the order
	of the reference, and are referred to in the paper by the numbers
	in brackets such as \cite{hugh,KLH}.  Style the reference list
	according to the following examples.
	\\
	
	\noindent (1) Book
	
	\noindent1.  Hughes, T. J. R., {\it The Finite Element Method,
		Linear Static and Dynamic Finite Element  Analysis},
	Prentice-Hall, Engelwood Cliffs, NJ, 1987.
	
	\bigskip
	
	\noindent (2) Paper in a journal
	
	\noindent2.  Koh, H. M., Lee, H. S. and Haber, R. B., ``Dynamic
	crack propagation analysis using Eulerian-Lagrangian kinematic
	descriptions'', {\it Computational Mechanics}, Vol. 3, 1988, pp.
	141-155.
	
	\bigskip
	\noindent (3) Chapter in a book
	
	\noindent3.  Riedel, H., ``Nucleation of Creep Cavities/Basic
	Theories", Chapter 7, {\it Fracture at High Temperatures},
	Springer-Verlag, Berlin, 1987.
	
	\bigskip
	\noindent (4) Paper in Conference Proceedings
	
	\noindent4.  Lee, H.S. and Koh, H.M., ``A Moving-Grid Finite
	Element Method for the Prediction of Dynamic Crack Propagation in
	Brittle Materials", {\it Proc. of the Second International
		Conference on Computer Aided Assessment and Control of Localized
		Damage}, Vol. 2, pp 463-480, Southampton U.K., July 1992.
	
	%%
	%%
	%% 참고 문헌  문헌의 형식에 주의하시어 작성시 아래를 이용하시기 바랍니다.
	%% Use the follwoing format for Reference section
	%%
	\begin{thebibliography}{11}
		%Book
		\bibitem {hugh}{Hughes, T. J. R.},
		\newblock{\em The Finite Element Method, Linear Static and Dynamic Finite
			Element  Analysis}, Prentice-Hall, Engelwood Cliffs, NJ, 1987.
		%
		%Paper in a journal
		\bibitem {KLH}{ Koh, H. M., Lee, H. S. and Haber, R. B.},
		{``Dynamic crack propagation analysis using Eulerian-Lagrangian
			kinematic descriptions''}, {\em Computational Mechanics}, Vol. 3,
		1988, pp. 141-155.
		%
		%Chapter in a book
		\bibitem {riedel}{Riedel, H.},
		{``Nucleation of Creep Cavities/Basic Theories"},  Chapter 7, {\it
			Fracture at High Temperatures}, Springer-Verlag, Berlin, 1987.
		%
		%Paper in Conference Proceedings
		\bibitem {LK}{Lee, H.S. and Koh, H.M.},
		{``A Moving-Grid Finite Element Method for the Prediction of
			Dynamic Crack Propagation in Brittle Materials"}, {\it Proc. of
			the Second International Conference on Computer Aided Assessment
			and Control of Localized Damage}, Vol. 2, pp 463-480, Southampton
		U.K., July 1992.
	\end{thebibliography}
\end{document}